\documentclass[a4paper,fleqn,10pt]{article}
\usepackage{standard}
\usepackage{multicol}
\geometry{top=30mm,bottom=30mm,left=30mm,right=30mm}
\begin{document}
  \hrule
  \medskip
  \begin{center}
    \Large
    \textbf{Kosmologie: Übungsserie 1}
  \end{center}
  \medskip
  \begin{minipage}[t]{0.45\textwidth}
    Markus Pawellek
  \end{minipage}
  \hfill
  \begin{minipage}[t]{0.45\textwidth}
    \begin{raggedleft}
      markuspawellek@gmail.com\\
    \end{raggedleft}
  \end{minipage}
  \medskip
  \hrule
  \bigskip

  \section*{Aufgabe 1} % (fold)
  \begin{multicols}{2}
  \label{sec:aufgabe_1}
    \paragraph{(a):}
    Ein Teilchen der Masse $m\in\setReal^+$ bewege sich reibungsfrei und nur unter Einwirkung einer Zwangskraft, erzeugt durch die Zwangsbedingung $g$, auf der zweidimensionalen Kugeloberfläche $\mathscr{S}^2$ mit Radius $R\in \setReal^+$.
    \[
      \function{g}{\setReal^3}{\setReal}
      \separate
      g(x)\define R^2 - \norm{x}^2
    \]
    \begin{align*}
      \mathscr{S}^2
      \define &\set{x\in\setReal^3}{\norm{x} = R} \\
      = &\set{x\in\setReal^3}{g(x) = 0}
    \end{align*}
    Durch die folgende surjektive stetig differenzierbare Abbildung Φ wird ein System generalisierter Koordinaten, die die Zwangsbedingung $g$ automatisch erfüllen, konstruiert.
    \[
      \function{Φ}{[0,π]\times[0,2π)}{\setReal^3}
    \]
    \[
      Φ(ϑ,φ)\define R
      \begin{pmatrix}
        \sin ϑ \cos φ \\
        \sin ϑ \sin φ \\
        \cos ϑ
      \end{pmatrix}
    \]
    \[
      \mathrm{im}\roundBrackets{g\circ Φ} = \set{0}{}
    \]
    Die entsprechende Jacobi-Matrix $\jacobian{Φ}$ der Abbildung Φ ergibt sich dann wie folgt für alle $ϑ\in [0,π]$ und $φ\in[0,2π)$.
    \[
      \jacobian{Φ}(ϑ,φ) = R
      \begin{pmatrix}
        \cos ϑ \cos φ & -\sin ϑ \sin φ \\
        \cos ϑ \sin φ & \sin ϑ \cos φ \\
        -\sin ϑ & 0
      \end{pmatrix}
    \]
    Es sei nun $x$ die zweimal stetig differenzierbare Kurve, die das Teilchen in Bezug auf die Zeit im Inertialsystem zurücklegt, und $q$ die zweimal stetig differenzierbare Kurve, die das Teilchen im Konfigurationsraum zurücklegt.
    \[
      \function{x}{[t_1,t_2]}{\setReal^3}
    \]
    \[
      \function{q}{[t_1,t_2]}{[0,π]\times[0,2π)}
    \]
    \[
      x = Φ\circ q
    \]
    Für die kinetische Energie $T$ des Teilchens ergibt sich die folgende Formel.
    \begin{align*}
      T
      &= \frac{m}{2} \norm{\timeDerivative{x}}^2
      = \frac{m}{2}\norm{\roundBrackets{Φ\circ q}'}^2 \\
      &= \frac{m}{2}\norm{ \roundBrackets{\jacobian{Φ}\circ q} \cdot \timeDerivative{q} }^2 \\
      &= \frac{m}{2} \roundBrackets{ \transpose{\timeDerivative{q}} \boxBrackets{\roundBrackets{\transpose{\jacobian{Φ}}\jacobian{Φ}}\circ q} \timeDerivative{q} } \\
      &= \frac{m}{2} \roundBrackets{g_{mn}\circ q} \cdot \timeDerivative{q}^m \timeDerivative{q}^n
    \end{align*}
    Der metrischen Tensor $\roundBrackets{g_{mn}}$ lässt sich direkt berechnen.
    \[
      \roundBrackets{g_{mn}}_{m,n\in\set{1,2}{}}(ϑ,φ) =
      \begin{pmatrix}
        R^2 & 0 \\
        0 & R^2\sin^2 ϑ
      \end{pmatrix}
    \]
    Explizit kann damit die kinetische Energie wie folgt beschrieben werden.
    \[
      T = \frac{mR^2}{2}\roundBrackets{\timeDerivative{ϑ}^2 + \timeDerivative{φ}^2 \sin^2ϑ}
    \]
    \[
      ϑ \define q^1
      \separate
      φ \define q^2
    \]
    Die dem Teilchen entsprechende Lagrange-Funktion kann nun nach dieser Betrachtung formuliert werden.
    \[
      \function{L}{\setReal\times\setReal^2\times\setReal^2}{\setReal}
    \]
    \[
      L\roundBrackets{t,ϑ,φ,{\timeDerivative{ϑ},\timeDerivative{φ}}} \define \frac{mR^2}{2}\roundBrackets{\timeDerivative{ϑ}^2 + \timeDerivative{φ}^2\sin^2ϑ}
    \]
    Die Wirkung $S$ wird standardmäßig durch ein Integral über die Lagrange-Funktion definiert.
    \[
      \function{S}{\mathrm{C}^2[t_1,t_2]}{\setReal}
    \]
    \[
      S(q) \define \integral{t_1}{t_2}{ L(t,q(t),\dot{q}(t)) }{t}
    \]
    Für eine stationäre Lösung der Wirkung $S$, formulieren wir die äquivalenten Euler-Lagrange-Gleichungen für $m\in\set{1,2}{}$.
    \[
      \leibnizDerivative{}{t}\partial_{m+3} L(t,q(t),\dot{q}(t)) = \partial_{m+1} L(\cdot,q,\dot{q})
    \]
    In expliziter Form lassen sie sich wie folgt aufschreiben.
    Dabei wurde bereits durch den Term $mR^2/2$ geteilt.
    \[
      \ddot{ϑ} - \dot{φ}^2\sinϑ\cosϑ = 0
    \]
    \[
      \ddot{φ}\sin^2ϑ + 2\dot{φ}\dot{ϑ}\sinϑ\cosϑ = 0
    \]
    Diese Gleichungen formen wir nun in die Geodätengleichungen um, damit das Ablesen der Christoffel-Symbole möglich wird.
    Hierbei wollen wir die Fälle $ϑ=0$ und $ϑ=π$ ignorieren.
    \[
      \ddot{ϑ} - \dot{φ}^2\sinϑ\cosϑ = 0
    \]
    \[
      \ddot{φ} + 2\dot{φ}\dot{ϑ}\cot ϑ = 0
    \]
    Die Christoffel-Symbole lassen sich nun leicht durch Ablesen bestimmen und in folgender Form notieren.
    \[
      \roundBrackets{Γ^1_{mn}}_{m,n\in\set{1,2}{}}(ϑ,φ) =
      \begin{pmatrix}
        0 & 0 \\
        0 & -\frac{1}{2}\sin 2ϑ
      \end{pmatrix}
    \]
    \[
      \roundBrackets{Γ^2_{mn}}_{m,n\in\set{1,2}{}}(ϑ,φ) =
      \begin{pmatrix}
        0 & \cot ϑ \\
        \cot ϑ & 0
      \end{pmatrix}
    \]

    Für einen beliebigen Längenkreis auf der Kugeloberflächen gilt nun $\dot{φ} = 0$.
    Wenden wir dies auf die Geodätengleichungen an, so erhalten wir sofort $\ddot{ϑ}=0$.
    Die Funktion ϑ lässt sich demnach wie folgt für gewisse Konstanten $ϑ_0\in [0,π]$ und $a\in \setReal$ und alle $t\in[t_1,t_2]$ notieren.
    \[
      ϑ(t) = ϑ_0 + at
    \]
    Dabei handelt es sich aber gerade um einen beliebigen Ausschnitt eines Längenkreises.
    Insbesondere ist damit auch jeder gesamte Längenkreis eine Lösung der Geodätengleichungen.

    Für einen beliebigen Breitenkreis gilt $\dot{ϑ} = 0$.
    Auch diese Gleichung wenden wir wieder auf die Geodätengleichungen an.
    \[
      \dot{φ}^2\sin 2ϑ = 0
      \separate
      \ddot{φ} = 0
    \]
    Für den Fall das $\dot{φ} = 0$ gilt, ergibt sich für die Lösung ein einziger Punkt auf der Kugeloberfläche.
    Der übrige Fall $\sin2ϑ = 0$ ist nur für $ϑ\in\set{0,\frac{π}{2},π}{}$ erfüllt.
    Auch hier ergeben sich für $ϑ\in\set{0,π}{}$ wieder einzelne Punkte als Lösung.
    Für $ϑ=\frac{π}{2}$ erhalten wir jedoch analog zu den Längenkreisen die einfache Bedingung $\ddot{φ}=0$.
    Die Lösung stellt also einen beliebigen Ausschnitt des Äquators dar.
    Insbesondere ist damit der gesamte Äquator der einzige Breitenkreis, der eine Lösung der Geodätengleichungen darstellt.

    \paragraph{(b):}
    Die Polarkoordinaten können durch die folgende surjektive stetig differenzierbare Abbildung definiert werden.
    \[
      \function{Φ}{\setReal\times\setReal_0^+\times[0,2π)}{\setReal^3}
    \]
    \[
      Φ(ct,r,φ)\define
      \begin{pmatrix}
        ct \\
        r \cosφ \\
        r \sinφ
      \end{pmatrix}
    \]
    Die Ableitung oder auch Jacobi-Matrix dieser Abbildung kann wie folgt für alle $t\in\setReal$, $r\in\setReal^+_0$ und $φ\in[0,2π)$ notiert werden.
    \[
      \jacobianΦ(ct,r,φ) =
      \begin{pmatrix}
        1 & 0 & 0 \\
        0 & \cosφ & -r\sinφ \\
        0 & \sinφ & r\cosφ
      \end{pmatrix}
    \]
    Weiterhin wollen wir die Minkowski-Metrik dieser flachen Raumzeit einführen.
    \[
      η \define \roundBrackets{η_{mn}} \define \roundBrackets{η^{mn}} \define
      \begin{pmatrix}
        -1 & 0 & 0 \\
        0 & 1 & 0 \\
        0 & 0 & 1
      \end{pmatrix}
    \]
    Der metrische Tensor kann nun mithilfe dieser Abbildungen beschrieben werden.
    \[
      \roundBrackets{g_{mn}}_{m,n}
      = \transpose{\jacobianΦ}η\jacobianΦ
      = \roundBrackets{η_{pq} \partial_mΦ^p \partial_nΦ^q}_{m,n}
    \]
    \[
      \roundBrackets{g_{mn}}_{m,n\in\set{0,1,2}{}} (ct,r,φ) =
      \begin{pmatrix}
        -1 & 0 & 0 \\
        0 & 1 & 0 \\
        0 & 0 & r^2
      \end{pmatrix}
    \]
    Um nun das eigentliche Problem zu lösen, wollen wir zunächst eine kleine Nebenrechnung durchführen, um eine andere Form der Christoffel-Symbole zu erhalten.
    \[
      Γ^i_{mn} = \frac{1}{2}g^{ik}\roundBrackets{
        \partial_m g_{nk} + \partial_n g_{mk} - \partial_k g_{mn}
      }
    \]
    Zunächst bilden wir die partiellen Ableitungen des metrischen Tensors.
    \[
      \partial_k g_{mn} = η_{pq}\roundBrackets{\partial_k\partial_m Φ^p \partial_n Φ^q + \partial_m Φ^p \partial_k\partial_n Φ^q}
    \]
    Diese werden wir, wie in den Christoffel-Symbolen angegeben, zusammenführen.
    % \[
    %   \partial_m g_{nk} + \partial_n g_{mk} - \partial_k g_{mn} =
    %   η_{pq}\roundBrackets{
    %     \partial_m\partial_n Φ^p \partial_k Φ^q + \partial_n Φ^p \partial_m\partial_k Φ^q
    %     + \partial_n\partial_m Φ^p \partial_k Φ^q + \partial_m Φ^p \partial_n\partial_k Φ^q
    %     -\partial_k\partial_m Φ^p \partial_n Φ^q - \partial_m Φ^p \partial_k\partial_n Φ^q
    %   }
    % \]
    \begin{align*}
      &\partial_m g_{nk} + \partial_n g_{mk} - \partial_k g_{mn} \\
      &= η_{pq}(
        \begin{aligned}[t]
          &\partial_m\partial_n Φ^p \partial_k Φ^q + \partial_n Φ^p \partial_m\partial_k Φ^q \\
          + &\partial_n\partial_m Φ^p \partial_k Φ^q + \partial_m Φ^p \partial_n\partial_k Φ^q \\
          - &\partial_k\partial_m Φ^p \partial_n Φ^q - \partial_m Φ^p \partial_k\partial_n Φ^q)
        \end{aligned} \\
      &=
        \begin{aligned}[t]
          &2η_{pq} \partial_m\partial_n Φ^p \partial_k Φ^q \\
          &+ η_{pq}( \partial_n Φ^p \partial_m\partial_k Φ^q - \partial_k\partial_m Φ^p \partial_n Φ^q )
        \end{aligned} \\
      &= 2η_{pq} \partial_k Φ^q \partial_m\partial_n Φ^p
    \end{align*}
    Die Jacobi-Matrix der gegebenen Polarkoordinaten ist invertierbar mit $\jacobian\bar{Φ} \define \inverse{\jacobianΦ}$, sofern wir das Argument des Radius ungleich Null wählen.
    Durch diese Eigenschaft erhalten wir die folgende Formel.
    \[
      \roundBrackets{g^{mn}}_{m,n} = \jacobian\bar{Φ}η\transpose{\jacobian\bar{Φ}}
      = \roundBrackets{η^{pq}\partial_p \bar{Φ}^m \partial_q \bar{Φ}^n}_{m,n}
    \]
    Diese Resultate setzen wir nun in die Christoffel-Symbole ein und erhalten die folgende neue Form.
    \begin{align*}
      Γ^l_{mn}
      &= η^{ij} \partial_i\bar{Φ}^l \partial_j\bar{Φ}^k η_{pq} \partial_k Φ^q \partial_m\partial_n Φ^p \\
      &= η^{ij} \partial_i\bar{Φ}^l η_{pq} δ_j^q \partial_m\partial_n Φ^p \\
      &= η^{ij} η_{pj} \partial_i\bar{Φ}^l \partial_m\partial_n Φ^p \\
      &= δ^i_p \partial_i\bar{Φ}^l \partial_m\partial_n Φ^p \\
      &= \partial_i\bar{Φ}^l \partial_m\partial_n Φ^i
    \end{align*}
    Die Invertierbarkeit der Jacobi-Matrix erlaubt uns zudem das Folgende.
    \[
      \partial_m \partial_n Φ^k = Γ^i_{mn} \partial_i Φ^k
    \]
    Wir definierun nun γ als die Kurve des Lichts im Inertialsystem und $q$ als die Kurve des Lichtes in den gegebenen Polarkoordinaten.
    \[
      \function{γ}{\setReal}{\setReal^3}
      \separate
      \function{q}{\setReal}{\setReal\times\setReal^+_0\times[0,2π)}
    \]
    Beide Kurven sind zweimal stetig differenzierbar und verbunden durch die folgende Beziehung.
    \[
      γ \define Φ\circ q
    \]
    Demnach ergeben sich für die Ableitungen der Kurve γ die folgenden Terme.
    \begin{align*}
      γ' &= \roundBrackets{\jacobian{Φ}\circ q} \cdot q' \\
      γ'' &= \roundBrackets{\jacobian^2Φ\circ q} (q',q') + \roundBrackets{\jacobianΦ\circ q} \cdot q'' \\
      &=
        \begin{aligned}[t]
          &\left[ \roundBrackets{\partial_n\partial_m Φ^k\circ q} \cdot q'^m q'^n \right. \\
          &\left. + \roundBrackets{\partial_m Φ^k\circ q} \cdot q''^m \right]_k
        \end{aligned}
    \end{align*}
    Hier können wir die zweiten Ableitungen durch die Christoffel-Symbole beschreiben.
    \begin{align*}
      γ''^k &=
        \begin{aligned}[t]
          &\roundBrackets{Γ^i_{mn}\partial_i Φ^k\circ q} \cdot q'^m q'^n \\
          &+ \roundBrackets{\partial_m Φ^k\circ q} \cdot q''^m
        \end{aligned} \\
      &= \roundBrackets{\partial_m Φ^k\circ q} \underbrace{\boxBrackets{q''^m + \roundBrackets{Γ^m_{ij}\circ q} \cdot q'^i q'^j}}_{=0} \\
      &= 0
    \end{align*}
    Die angegebenen Ausdrücke müssen Null werden, da es sich bei den Termen der rechten Klammer um die linken Seiten der Geodätengleichungen im System der Polarkoordinaten handelt.
    Das Lichtsignal bewegt sich auf den Geodäten.
    Wir haben gezeigt, dass die Kurve des Lichts im Inertialsystem eine gleichförmig geradlinige Bewegung ausführt, da die Gleichung $γ'' = 0$ gilt.
    Die Kurve des Lichts muss zudem die konstante Geschwindigkeit $c$ aufweisen, da Licht die folgende Gleichung erfüllen muss.
    \[
      g_{mn}q'^m q'^n = 0
    \]

    Analog lässt sich diese allgemeine Betrachtung auch durch direktes Nachrechnen anhand der gegebenen Metrik zeigen.
    Wir definieren zunächst die Lagrange-Funktion und konstruieren die Geodätengleichungen.
    \[
      \function{L}{\setReal\times\roundBrackets{\setReal\times\setReal^+_0\times[0,2π)}^2}{\setReal}
    \]
    \[
      L(λ,q,q')\define g_{mn}(q) q'^m q'^n
    \]
    Die Euler-Lagrange-Gleichungen ergeben sich wie folgt, wobei $ct \define q^0$, $r\define q^1$ und $φ\define q^2$.
    Zudem müssen wir beachten, dass es sich um Licht handelt.
    \[
      r'' - rφ'^2 = 0
      \separate
      φ'' + \frac{2}{r}r'φ' = 0
    \]
    \[
      t'' = 0
      \separate
      r'^2 + r^2φ'^2 = c^2t'^2
    \]
    Nun berechnen wir die zweiten Ableitungen der Kurve im Inertialsystem.
    \begin{align*}
      γ''^1 &=
        \begin{aligned}[t]
          r''\cosφ &- r'φ'\sinφ - r'φ'\sinφ \\
          &- rφ''\sinφ - rφ'^2\cosφ
        \end{aligned} \\
      &=
        \begin{aligned}[t]
          &\roundBrackets{r'' - rφ'^2}\cosφ \\
          &- \roundBrackets{rφ'' + 2r'φ'}\sinφ
        \end{aligned} \\
      &= 0 \\
      γ''^2 &=
        \begin{aligned}[t]
          r''\sinφ &+ r'φ'\cosφ + r'φ'\cosφ \\
          &+ rφ''\cosφ - rφ'^2\sinφ
        \end{aligned} \\
      &=
        \begin{aligned}[t]
          &\roundBrackets{r'' - rφ'^2}\sinφ \\
          &+ \roundBrackets{rφ'' + 2r'φ'}\cosφ
        \end{aligned} \\
      &= 0
    \end{align*}
    Auch nach dieser Rechnung breitet sich Licht geradlinig aus.
    Für die Lösung der Differentialgleichungen verwenden wir zunächst die Geodätengleichung für $φ''$ und separieren $r$ und $φ$.
    \[
      \frac{φ''}{φ'} = -\frac{2r'}{r}
    \]
    Diese Gleichung lässt sich nun sehr leicht durch Integration und die Anwendung der Substitutionsregel lösen.
    \[
      \integral{λ_1}{λ_2}{\frac{φ''(λ)}{φ'(λ)}}{λ} = \integral{φ'(λ_1)}{φ'(λ_2)}{\frac{1}{s}}{s} = \ln \frac{φ'(λ_2)}{φ'(λ_1)}
    \]
    \[
      \integral{λ_1}{λ_2}{\frac{-2r'(λ)}{r(λ)}}{λ} = \integral{r(λ_1)}{r(λ_2)}{\frac{-2}{s}}{s} = \ln \frac{r^2(λ_1)}{r^2(λ_2)}
    \]
    Wir erhalten damit den folgenden Zusammenhang für Konstanten $K,λ_0\in\setReal$.
    \[
      φ' = \frac{K}{r^2} \define \frac{φ'(λ_0)r^2(λ_0)}{r^2}
    \]
    Die Geodätengleichung für $t''$ sagt uns, dass es sich bei $t'$ um eine Konstante handelt.
    Diese beiden Zusammenhänge wenden wir nun auf die Geodätengleichung für $r''$ und das metrische Skalarprodukt an.
    \[
      r'' - \frac{K^2}{r^3} = 0
      \separate
      r'^2 + \frac{K^2}{r^2} = c^2t'^2
    \]
    Durch Ableiten der zweiten Gleichung sehen wir, dass jede Lösung der zweiten Gleichung auch eine Lösung der ersten Gleichung ist.
    \[
      2r'r'' - 2r'\frac{K^2}{r^3} = 0
    \]
    Das Lösen der zweiten Gleichung ist durch Trennung der Variablen möglich.
    Wir stellen diese Gleichung zunächst um.
    \[
      \pm r' \boxBrackets{ c^2t'^2 - \frac{K^2}{r^2} }^{-\frac{1}{2}} = 1
    \]
    Jetzt können wir durch Integrieren und Verwendung der Substitutionsregel die Differentialgleichung lösen.
    Es sei dafür $λ\in\setReal$.
    \begin{align*}
      &\pm(λ-λ_0) \\
      &= \integral{λ_0}{λ}{ r'(s) \boxBrackets{ c^2t'^2 - \frac{K^2}{r^2(s)} }^{-\frac{1}{2}} }{s} \\
      &= \integral{r(λ_0)}{r(λ)}{ \roundBrackets{ c^2t'^2 - \frac{K^2}{s^2} }^{-\frac{1}{2}} }{s} \\
      &= \integral{r(λ_0)}{r(λ)}{ \frac{s}{\sqrt{c^2t'^2s^2 - K^2}} } {s} \\
      &= \frac{\sqrt{c^2t'^2r^2(λ) - K^2}}{c^2t'^2} + C
    \end{align*}
    \[
      C \define - \frac{\sqrt{c^2t'^2r^2(λ_0) - K^2}}{c^2t'^2}
    \]
    Durch Umstellen nach $r(λ)$ erhalten wir den folgenden Ausdruck.
    \[
      r(λ) = \pm ct' \sqrt{\boxBrackets{\pm(λ-λ_0)-C}^2 + \frac{K^2}{c^4t'^4}}
    \]
    Durch Verwendung des gezeigten Zusammenhangs zwischen $r$ und $φ'$ lässt sich nun $φ'$ direkt notieren.
    \[
      φ'(λ) = \frac{K}{c^2t'^2\boxBrackets{\pm(λ-λ_0)-C}^2 + \frac{K^2}{c^2t'^2}}
    \]
    Auch diese Gleichung lässt sich integrieren und man erhält das Folgende für eine Konstante $D\in\setReal$.
    \[
      φ(λ) = \arctan\boxBrackets{\frac{c^2t'^2\roundBrackets{λ-λ_0-C}}{K}} + D
    \]
    % \[
    %   D\define \arctan\roundBrackets{\frac{c^2t'^2C}{K}}
    % \]
  % section aufgabe_1 (end)
  \end{multicols}

  \newpage

  \section*{Aufgabe 2} % (fold)
  \label{sec:aufgabe_2}
  \begin{multicols}{2}
    \paragraph{(a):}
    Die Geodäten-Gleichungen können im Allgemeinen wie im Folgenden gezeigt mithilfe der Christoffel-Symbole notiert werden.
    \[
      q''^k + Γ^k_{mn} q'^m q'^n = 0
    \]
    Handelt es sich nun bei $q^α$ um konstante Funktionen und damit um ein mitbewegtes Koordinatensystem, so müssen die folgenden Aussagen gelten.
    \[
      q'^{α} = q''^{α} = 0
    \]
    Die Geodätengleichungen vereinfachen sich damit zu den folgenden Gleichungen.
    \[
      Γ^{α}_{00}\roundBrackets{q'^0}^2 = 0
      \separate
      \roundBrackets{q^0}'' + Γ^0_{00} \roundBrackets{q'^0}^2 = 0
    \]
    In der Robertson-Walker-Geometrie kennen wir bereits die Christoffel-Symbole und können diese nutzen, um die Gleichungen weiter zu vereinfachen.
    \[
      Γ^k_{00} = 0 \quad\implies\quad ct'' = 0
    \]
    Damit wurde gezeigt, dass im System der mitbewegten Koordinaten einer Galaxie diese sich in geodätischer Bewegung befindet.

    Die Geodätengleichungen lauten in einer nicht-indexbasierten Schreibweise wie folgt.
    \begin{align*}
      0 &=
        \begin{aligned}[t]
          &r'' + \frac{εr}{1-εr^2}r'^2 - r(1-εr^2)ϑ'^2 \\
          &- r(1-εr^2)\sin^2ϑ φ'^2 + \frac{2a'\circ t}{a\circ t}r't'
        \end{aligned}\\
      0 &=
        \begin{aligned}[t]
          &ϑ'' + \frac{2}{r}r'ϑ' -\sinϑ\cosϑ φ'^2 \\
          &+ \frac{2a'\circ t}{a\circ t}ϑ't'
        \end{aligned} \\
      0 &=
        \begin{aligned}[t]
          &φ'' + \frac{2}{r}r'φ' + 2\cotϑ ϑ'φ' \\
          &+ \frac{2a'\circ t}{a\circ t}φ't'
        \end{aligned} \\
      0 &=
        \begin{aligned}[t]
          &c^2t'' + \frac{(aa')\circ t}{1-εr^2}r'^2 + ((aa')\circ t)r^2 ϑ'^2 \\
          &+ ((aa')\circ t)r^2\sin^2ϑφ'^2
        \end{aligned}
    \end{align*}
    Bewegt sich ein Teilchen nur in radialer Richtung, so gilt $ϑ' = φ' = 0$.
    Dies setzen wir nun in die Geodätengleichungen ein und erhalten das Folgende.
    \begin{align*}
      0 &= r'' + \frac{εr}{1-εr^2}r'^2 + \frac{2a'\circ t}{a\circ t}r't' \\
      0 &= c^2t'' + \frac{(aa')\circ t}{1-εr^2}r'^2
    \end{align*}
    Die resultierenden Differentialgleichungen hängen nur noch von $r$ und $t$ ab.
    Offensichtlich stellt damit eine reine radiale Bewegung eine Lösung der Geodätengleichungen dar.

    Bewegt sich ein Teilchen allerdings nur anfangs in radialer Richtung, so gelten $ϑ'(λ_0) = φ'(λ_0) = 0$ und $ϑ(λ_0) = ϑ_0\in[0,π]$ und $φ(λ_0)=φ_0\in [0,2π)$ für einen Parameter $λ_0\in\setReal$.
    Damit ist nun allerdings ein Anfangswertproblem gegeben.
    Nach dem Satz von Picard-Lindelöf ist die erhaltene Lösung dieses Differentialgleichungssystems jedoch global eindeutig.
    Die gerade beschriebene Lösung erfüllt die Anfangswerte und ist damit umgekehrt die einzige mögliche Lösung.
    Ein Teilchen welches sich anfangs in radialer Richtung fortbewegt, behält damit diese Bewegungsrichtung auch bei.

    \paragraph{(b):}
    Es sei ein unbegrenzt expandierender Robertson-Walker-Kosmos gegeben.
    Damit ist $a$ eine stetig differenzierbare strikt monoton wachsende Funktion, welche nicht beschränkt ist.
    \[
      \function{a}{\setReal}{\setReal^+}
      \separate
      a' > 0
      \separate
      \lim_{t\converges\infty}a(t) = \infty
    \]
    Es sei $\function{q}{[λ_0,λ_1]}{\setReal^4}$ die Kurve des in Bewegung gesetzten Teilchens.
    Dann folgt aus der radialen Geschwindigkeit und den gezeigten Aussagen aus Aufgabenteil (a), dass die folgenden Gleichungen gelten müssen, wobei $ct\define q^0$ und $r\define q^1$.
    \begin{align*}
      0 &= r'' + \frac{εr}{1-εr^2}r'^2 + \frac{2a'\circ t}{a\circ t}r't' \\
      0 &= c^2t'' + \frac{(aa')\circ t}{1-εr^2}r'^2
    \end{align*}
    Um eine valide Kurve des Teilchens zu beschreiben, muss die Zeitkurve strikt monoton wachsen.
    Nach der zweiten Gleichung ist die zweite Ableitung $t''$ jedoch negativ.
    \[
      c^2t'' = -\frac{(aa')\circ t}{1-εr^2}r'^2
    \]
    Um zu verhindern, dass es ein $λ\in\setReal$ gibt, sodass $t'(λ) \leq 0$, muss das Folgende gelten.
    \[
      \lim_{λ\converges \infty} t''(λ) = 0 = \lim_{λ\converges \infty}\frac{(aa')\circ t(λ)}{1-εr^2(λ)}r'^2(λ)
    \]
    Diese Aussage ist äquivalent zu der folgenden Aussage, da der Faktor vor $r'^2$ ständig positiv ist.
    \[
      \lim_{λ\converges \infty}r'(λ) = 0
    \]
    Das Teilchen muss also im Galaxiensubstrat in Grenzwert zur Ruhe kommen.

    Es sei nun $ε=0$.
    In diesem Falle vereinfachen sich die gezeigten Geodätengleichungen noch weiter.
    \[
      r'' + \frac{2a'\circ t}{a\circ t}r't' = 0
    \]
    \[
      c^2t'' + ((aa')\circ t)r'^2 = 0
    \]
    Wir werden die erste Gleichung verwenden und die Variablen $r$ und $t$ separieren.
    \[
      \frac{r''}{r'} = - \frac{2a'\circ t}{a\circ t}t'
    \]
    Beide Seiten lassen sich nun durch die Anwendung der Substitutionsregel integrieren.
    \[
      \integral{λ_0}{λ}{ \frac{r''(s)}{r'(s)} }{s} = \ln \frac{r'(λ)}{r'(λ_0)}
    \]
    \[
      \integral{λ_0}{λ}{ - \frac{2a'\circ t}{a\circ t}t' }{s} = \integral{t(λ_0)}{t(λ)}{ -\frac{2a'(s)}{a(s)} }{s}
    \]
    \[
      = \integral{a(t(λ_0))}{a(t(λ))}{ -\frac{2}{s} }{s} = \ln \frac{a^2(t(λ_0))}{a^2(t(λ))}
    \]
    Durch Gleichsetzen der beiden Integrale entsteht der folgende Zusammenhang zwischen $r'$ und $t$.
    \[
      r'(λ) = \frac{r'(λ_0)a^2(t(λ_0))}{a^2(t(λ))} \reverseDefine \frac{K}{a^2(t(λ))}
    \]
    Diesen Zusammenhang wollen wir nun in der zweiten Gleichung anwenden.
    \[
      c^2t'' + K^2\frac{a'\circ t}{a^3\circ t} = 0
    \]
    Wir multiplizieren sie mit $t'$, um eine Integration zu ermöglichen.
    \[
      c^2t't'' + K^2 t'\frac{a'\circ t}{a^3\circ t} = 0
    \]
    Auch hier verwenden wir bei der Integration wieder die Substitutionsregel.
    \[
      \frac{c^2t'^2}{2} - \frac{K^2}{2a^2\circ t} = \frac{c^2t'^2(λ_0)}{2} - \frac{K^2}{2a^2(t(λ_0))} \reverseDefine \frac{C}{2}
    \]
    Durch Umstellen dieser Gleichung erhalten wir den folgenden Ausdruck.
    \[
      \pm ct' \roundBrackets{C + \frac{K^2}{a^2\circ t}}^{-\frac{1}{2}} = 1
    \]
    Auch hier lässt sich mithilfe der Substitutionsregel die Gleichung integrieren.
    \[
      \integral{λ_0}{λ}{ ct'(s) \roundBrackets{C + \frac{K^2}{a^2(t(s))}}^{-\frac{1}{2}} }{s} = \pm(λ-λ_0)
    \]
    \[
      = \integral{t(λ_0)}{t(λ)}{c\roundBrackets{C + \frac{K^2}{a^2(s)}}^{-\frac{1}{2}}}{s}
    \]
    Zum Schluss erhalten wir durch den Zusammenhang von $r'$ und $t$ noch die folgende Gleichung.
    \[
      r(λ) = r(λ_0) + \integral{λ_0}{λ}{ \frac{K}{a^2(t(s))} }{s}
    \]
    Offensichtlich muss sich der Punkt hält das Teilchen bei der radialen Koordinate $r_\infty$.
    \[
      r_\infty \define \lim_{λ\converges \infty}r(λ)
    \]
    \[
      r_\infty = r(λ_0) + \integral{λ_0}{\infty}{ \frac{K}{a^2(t(s))} }{s}
    \]
    % \[
    %   \roundBrackets{r\circ\inverse{t}}' = \frac{r'\circ\inverse{t}}{t'\circ\inverse{t}}
    % \]
    % \[
    %   \roundBrackets{r\circ\inverse{t}}'' = \frac{r''\circ\inverse{t} + \roundBrackets{r'\circ \inverse{t}}\roundBrackets{t''\circ\inverse{t}}}{\roundBrackets{t'\circ\inverse{t}}^2}
    % \]
  \end{multicols}
  % section aufgabe_2 (end)

\end{document}